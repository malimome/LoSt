\documentclass[12pt]{article}
\usepackage{amssymb}

\newtheorem{theorem}{Theorem}[section]
\newtheorem{corollary}{Corollary}[section]%[thm]
\newtheorem{lemma}{Lemma}[section]%
\newtheorem{proposition}{Proposition}[section]
\newtheorem{definition}{Definition}[section]
\newtheorem{remark}{Remark}[section]
\newtheorem{example}{Example}[section]
\newtheorem{conjecture}{Conjecture}[section]
\newenvironment{proof}{\noindent {\em {Proof}}.}{$\square$\medskip}
\newcommand{\property}[1]{\newline \noindent {\bf \emph{#1}.}}

\def\nor{\underline{\triangleleft}\ }
\def \im {\Longrightarrow }
\def \sn {\triangleleft \triangleleft}
\def \din {\indent\indent}
\def \dqu {\quad\quad}

\newcommand{\ret}{\Longrightarrow}
\newcommand{\D}{\mathcal{D}}
\newcommand{\M}{\mathcal{M}}
\newcommand{\Fr}{\mathcal{F}}
\newcommand{\Zn}{\mathbb{Z}/n\mathbb{Z}}
\newcommand{\N}{\mathbb{N}}
\newcommand{\ZZ}[1]{\mathbb{Z}/#1\mathbb{Z}}
\newcommand{\abs}[1]{\left\vert#1\right\vert}
\newcommand{\set}[1]{\left\{#1\right\}}
\newcommand{\seq}[1]{\left<#1\right>}
\newcommand{\norm}[1]{\left\Vert#1\right\Vert}
\newcommand{\essnorm}[1]{\norm{#1}_{\ess}}

\title{LoST protocol}
\author{Implementation details}


\begin{document}
\maketitle

\section{LoST protocol}
A cloud server claims to have a file stored in its repository located in a certain region. 
Our aim is to design a protocol, called LoST, that can verify that the file is residing in the 
claimed region. The problem of verifying whether a server have a certain file or not, 
is called proof of retrieval or POR in brief.
To design LoST, we use a combination of POR protocol in \cite{POR}, and computing round trip time (RTT).
In POR, an encoding of a file is sent to the cloud server from the user and then the user 
challenge responses the cloud server to verify that the file is actually stored in the server. 
Computing RTT is a method that estimates the location of a server, usually in a wireless network, by 
sending a number of packets from different known locations and measuring the RTT of 
the packets. The devices that send these packets to the server are called landmarks.
Therefore, LoST uses RTT to estimate the location of the cloud server that has stored a 
file, but instead of sending arbitrary packets as in RTT, landmarks challenge responses the cloud server 
using POR, to verify that the server has stored the claimed file. 

\subsection{Setup}
In our protocol, by challenge responding the cloud server, we mean that the 
POR scheme in \cite{POR} is used to verify the file content. Therefore our 
protocol is actually an RTT protocol over Internet (not a wireless network) 
using the challenge response in POR. 

We restrict our experiment for 30 landmarks spread over North America and Europe. 
Then for a cloud server in these regions, we use about 10 landmarks to estimate the 
location of cloud server. For wireless networks, usually 3 landmarks are used to estimate 
the server location, but since Internet is not as reliable as a wireless network, we use 
10 landmarks for one cloud server to output a region with smaller boundaries. 

The landmarks are carefully selected from the planet lab nodes. We selected the nodes that 
had better computation times and their up time was more reliable. 

\subsection{LoST phases}
The LoST protocol has three phases:
\begin{itemize}
\item Learning phase: The protocol is run for all landmarks and cloud servers. The RTT 
is measured and based on the measured times and the actual distance of the landmark from the server, 
a line is estimated that can transform a measured time to distance. 
We use the best fit line for this purpose. The line is calculated for three different challenge sizes. 

\item Measure time: A cloud server claims to have a file in a certain region. 
Then a number of landmarks are selected based on the location of the server. Then the landmarks 
challenge response the cloud server and output their times. The times are transformed 
to distances using the best fit lines from the previous phase.

\item Region estimation: Using the distances from the previous phase, we have a number of 
landmarks with known locations and their distance to the cloud server. Now we use a method called 
trilaterarion to estimate the location (or region) of the cloud server.
\end{itemize}

\noindent{\bf Learning phase.}
GAVEN did this part.\\

\noindent{\bf Measure time.}
To estimate the position of the server, a set of landmarks are chosen to
challenge response the server and compute the RTT. For each landmark, we 
run the protocol a few times (here n? times). Then the (minimum) time is
converted into distance using the time-distance line from the learning
phase. \\

\noindent{\bf Region estimation.}
Therefore we have a set of nodes with known location and their distance to the target server. 
We can abstract this problem to a geometric problem, called trilateration. 
Trilateration is the process of finding the location of a unknown point when its 
distance to some known points is given. Trilateration estimates the centre
point of each set of three landmarks with given distance from the centre
point. We can summarize the problem as follows:\\

\noindent{\bf Problem:} We have the distance of $t$ landmarks (with known location) to a server 
with unknown location.\\
\noindent{\bf Solution:} An estimation of the server location. The output can be a centre point or 
a region. \\

Here we are interested in locating the position 
of a server which is residing on earth surface. We actually assume that the elevation 
of the server on earth is zero. 

This problem was investigated in GPS systems where a few GPS antennas want to estimate location 
of an object by measuring the RTT of the signal sent to the GPS device. This measurement 
is converted to distance and then trilateration is used to estimate the location of the GPS device.
Our setting is different from GPS which works on wireless communication whereas our protocol 
is on wired networks. We also do not consider elevation of the server. \\

\noindent{\bf Our protocol.}
Our protocol applies the appropriate transformations needed for trilateration on
earth and then computes the centre point using Trilateration. Having a number
of centre points, a circle is drawn around each of them and the number
of centre points in the circle are counted. The output would be the
centre point which has the highest number of centre points in its
circle. We will discuss the ideal algorithm for region estimation in next section. 


\subsection{Real World implementation problems and improvement ideas}
The errors in our protocol comes from all three phases. \\

\subsubsection{Learning phase error} 
For the learning phase, we expected that the distance should not change for 
different challenge sizes. This means that the best fit lines for different 
challenge sizes must be parallel. However because of random behaviours in 
computation time of the server and network delay (caused by routing, packet size, 
and packet break down issues) we did not get completely parallel lines. 

\subsubsection{Measure Time error}
The same network and computation fluctuations from previous phase effected 
the times measured in this phase. \\

We have the following conclusions from the experiment for the first two phases:
\begin{itemize}
\item To measure the time, the POR is run for a number of times and then 
we use the minimum time. We used minimum because the minimum time was more 
consistent over different period of times that we ran the experiment and measure the time. 
The average is always biased by some abnormal cases that effects the average time. For example, 
we had computation times in the servers, that were even 1000,000 times more than the normal 
computation time. We also tried the median, but minimum times were more consistent during different 
period of time.

\item Inaccuracy of the timing is a property of Planet Lab nodes, so 
we do not expect these to exist in real landmarks. This inaccuracy comes 
mostly from computation time fluctuations rather than network delays, so if 
the server can guarantee a consistent computation time, we would expect better results. 
\end{itemize}

\noindent{\bf Improvement.} To remove the computation time effect, we can have one landmark run the 
POR and verifies the file existence. Then other landmarks simply send a message to the server 
and compute the RTT. This will eliminate the computation time fluctuations as 
no landmark that is selected to measure the time will run the POR. 

Also a good mechanism to filter the abnormal measurements may help reduce the errors in this phase.

\subsubsection{Region estimation error}
In this phase, we have a circle around the landmarks and we expect the right region for 
the server is the region that is inside all circles, or more circles. So the best estimation 
is the region with highest opacity (density of circles). 

In this case, we have a system of non linear circle inequalities and we want to find
the region with highest opacity (density):
\[
(x-a_i)^2 + (y-b_i)^2 \le r_i^2, i=1 \rightarrow \#circles.
\]
Note that the intersection area for all the circles may be null because 
of error in time (or line) computation (the system may not have a solution) 
so we are looking for the highest
density region, the region that is contained in most circles.

However as implementing this algorithm was hard for more than 3 circles, we implemented an 
approximation of this algorithm that find the centre points. We expect that the centre point that 
we find is inside the region that has the highest opacity inside the circles. Therefore, our method 
is a good approximation of the server location. But calculating the highest opacity region 
is a more reliable method as it is closer to our intuition for the region that server resides in.
This may introduce some error in our computations. \\

\noindent{\bf Improvement.} The error in this phase can be reduced by the following changes:
\begin{itemize}
\item If we consider the minimum time, we get a circle around each landmark. One improvement can be 
considering minimum and median times (best fit lines given from minimum and median times) 
to get two distances. This will give us a ring around each landmark rather than a circle. 
However finding the intersection of rings can be even harder than circles.

\item Selecting the landmarks carefully around the server. For now, we choose the 10 closest 
landmarks around the server. But an improvement can be to consider the landmarks around the server 
such that they make a shape close to a circle.

\item Find the best number for the landmarks around the server. Is 10 the optimum case? 
In general, a good algorithm to select the landmarks around the server may reduce the error.

\item Our method of finding the centre point can also be improved. We can output the three 
centre points with highest number of centre points inside. In this case, we expect to server 
to be close to the triangle that these three centre points make.
\end{itemize}

\subsection{Implementation languages}
We implemented the main POR protocol in C/C++. The experiment is run through a number of 
bash scripts and the final region of the server is calculated by a python script. The python 
script implements the third phase, i.e. trilateration and estimation of the centre point.

\bibliographystyle{alpha}
\bibliography{momeni}
\end{document}
